\documentclass{article}
\usepackage[utf8]{inputenc}
\usepackage[polish]{babel}
\usepackage{graphicx}
\usepackage[T1]{fontenc}
\usepackage{wrapfig}
\usepackage{fancyhdr}
\usepackage{tabularx}




%\usepackage[a4paper,top=2cm,bottom=2cm,left=3cm,right=3cm,marginparwidth=1.75cm]{geometry}
\usepackage[includeheadfoot,
            left=1in,
            right=1in,
            top=1cm,
            bottom=1cm,
            headheight=1cm]{geometry}


\pagestyle{fancy}
\fancyhf{}  
\renewcommand{\headrulewidth}{0pt}  
\renewcommand{\footrulewidth}{0pt}  

\def\justifying{%
  \rightskip=0pt
  \spaceskip=0pt
  \xspaceskip=0pt
  \relax
}

\fancypagestyle{firstpage}
{
    \fancyhead[L]       % <-----------------LEWA STRONA-----------------
        {
     
            \vspace{0.25cm}            
            \textbf{\textit{Miłosz Grześkowiak}} \\
            {\textit{Rok 1/S2 - Informatyka Stosowana i Systemy Pomiarowe}} \\
            {\textit{06 Marca 2025}} \\
        }

        
    \fancyhead[R]   % <---------------------PRAWA STRONA------------
        {
         
             \vspace{0.25cm}
             \textit{06 Marca 2025}  \\ 
             {Prowadząca: } \\ 
             {\textit{dr. Sylwia Owczarek}} \\  
        
        }
}
\begin{document}
\textbf{ }\\
\textbf{ }\\
\thispagestyle{firstpage}
\centering
\section*{Ćwiczenie nr. 1}
\subsection*{Temat: Dokładność pomiaru długości}
\textbf{ }\\
\textbf{ }\\
\textbf{ }\\
\textbf{ }\\
\begin{tabularx}{1\textwidth} { 
  | >{\centering\arraybackslash}X |     % 1  }
  | >{\centering\arraybackslash}X |     % 2  }
  | >{\centering\arraybackslash}X |
  | >{\centering\arraybackslash}X |
  | >{\centering\arraybackslash}X |
  | >{\centering\arraybackslash}X |
  | >{\centering\arraybackslash}X |
  | >{\centering\arraybackslash}X |
  | >{\centering\arraybackslash}X |% 3  }   LICZBA 
  | >{\centering\arraybackslash}X |}    % 4  }   KOLUMN
 \hline


 %---------------------------OPIS-------------------------------
 \# 
 & 1. długość boku a kostki mierzona linijką [mm]
 & 2. długość boku b kostki mierzona linijką [mm]
 & 3. długość boku c kostki mierzona linijką [mm]
 & 4. długość boku a kostki mierzona suwmiarką [mm]
 & 5. długość boku b kostki mierzona suwmiarką [mm]
 & 6. długość boku c kostki mierzona suwmiarką [mm]
 & 7. długość boku a kostki mierzona mikrometrem [mm]
 & 8. długość boku b kostki mierzona mikrometrem [mm]
 & 9. długość boku c kostki mierzona mikrometrem [mm] \\
%---------------------------OPIS-------------------------------

 
\hline
\hline
%---------------------------DANE-------------------------------
\hline 1 & 14 & 14 & 14 & 14.45 & 14.50 & 14.45 & 14.62 & 14.74 & 14.75 \\
\hline 2 & 14 & 14 & 14 & 14.50 & 14.45 & 14.40 & 14.64 & 14.76 & 14.63 \\
\hline 3 & 14 & 14 & 14 & 14.40 & 14.40 & 14.40 & 14.65 & 14.76 & 14.60 \\
%---------------------------DANE-------------------------------
\hline
\end{tabularx}

\textbf{ }\\


\raggedright
    {
        {Dokładność wartości z pomiaru 1-3} \\
        {$\Delta_1 = \pm1 mm$}\\
        \textbf{ }\\
        {Dokładność wartości z pomiaru 4-6} \\
        {$\Delta_2 = \pm0.05 mm$}\\
        \textbf{ }\\
        {Dokładność wartości z pomiaru 7-9} \\
        {$\Delta_3 = \pm0.01 mm$}\\
        \textbf{ }\\
    }

\begin{tabularx}{1\textwidth} { 
  | >{\centering\arraybackslash}X |     % 1  }
  | >{\centering\arraybackslash}X |     % 2  }
  | >{\centering\arraybackslash}X |
  | >{\centering\arraybackslash}X |
  | >{\centering\arraybackslash}X |}    % 4  }   KOLUMN
 \hline


 %---------------------------OPIS-------------------------------
 \# 
 & 1. promień zewnętrzny pierścienia mierzony linijką [mm]
 & 2. promień wewnętrzny pierścienia mierzony linijką [mm]
 & 3. promień zewnętrzny pierścienia mierzony suwmiarką [mm]
 & 4. promień wewnętrzny pierścienia mierzony suwmiarką [mm] \\
%---------------------------OPIS-------------------------------

 
\hline
\hline
%---------------------------DANE-------------------------------
\hline 1 & 15 & 9 & 15.30 & 9.00 \\
\hline 2 & 15 & 9 & 15.40 & 8.70 \\
\hline 3 & 15 & 9 & 15.50 & 9.00 \\
\hline 4 & 15 & 9 & 15.50 & 8.40 \\
\hline 5 & 15 & 9 & 15.45 & 9.00 \\
%---------------------------DANE-------------------------------
\hline
\end{tabularx}

\textbf{} \\

\raggedright
    {
        {Dokładność wartości z pomiaru 1-2} \\
        {$\Delta_1 = \pm1 mm$}\\
        \textbf{ }\\
        {Dokładność wartości z pomiaru 3-4} \\
        {$\Delta_2 = \pm0.05 mm$}\\
        \textbf{ }\\
    }

\begin{tabularx}{1\textwidth} { 
  | >{\centering\arraybackslash}X |     % 1  }
  | >{\centering\arraybackslash}X |     % 2  }
  | >{\centering\arraybackslash}X |}    % 3  }   KOLUMN
 \hline


 %---------------------------OPIS-------------------------------
 \# 
 & 1. wysokość walca mierzona suwmiarką [mm]
 & 2. średnica walca mierozona mikrometrem [mm] \\
%---------------------------OPIS-------------------------------

 
\hline
\hline
%---------------------------DANE-------------------------------
\hline 1 & 30.70 & 5.93 \\
\hline 2 & 30.60 & 5.95 \\
\hline 3 & 30.65 & 5.93 \\
\hline 4 & 30.65 & 5.93 \\
\hline 5 & 30.60 & 5.96 \\
%---------------------------DANE-------------------------------
\hline
\end{tabularx}

\textbf{} \\

\raggedright
    {
        {Dokładność wartości z pomiaru 1} \\
        {$\Delta_1 = \pm0.05 mm$}\\
        \textbf{ }\\
        {Dokładność wartości z pomiaru 2} \\
        {$\Delta_2 = \pm0.01 mm$}\\
        \textbf{ }\\
    }

\pagebreak


\centering

\section*{ZAGADNIENIA TEORETYCZNE}

\raggedright
    {1. Wprowadzenie do pomiarów fizycznych:

    - Pomiary fizyczne są podstawą eksperymentów naukowych i inżynierskich. Ich celem jest określenie wartości wielkości fizycznych z jak największą dokładnością.

    - Rodzaje pomiarów: bezpośrednie (np. pomiar długości linijką) i pośrednie (np. obliczanie objętości na podstawie wymiarów). \\
    2. Błędy pomiarowe:

    - Błąd systematyczny: Wynika z nieprawidłowego działania przyrządu pomiarowego lub metody pomiarowej. Może być spowodowany np. nieprawidłową kalibracją narzędzia.

    - Błąd przypadkowy: Wynika z niekontrolowanych czynników zewnętrznych, takich jak drgania, zmiany temperatury czy subiektywne odczyty.

    - Błąd gruby: Wynika z pomyłki eksperymentatora, np. błędnego odczytu skali. \\
    3. Narzędzia pomiarowe:

    - Linijka: Proste narzędzie do pomiaru długości z dokładnością do 1 mm. Stosowane do pomiarów o niskiej precyzji.

    - Suwmiarka: Narzędzie do pomiaru długości, średnic wewnętrznych i zewnętrznych z dokładnością do 0,05 mm.

    - Śruba mikrometryczna: Precyzyjne narzędzie do pomiaru długości z dokładnością do 0,01 mm. Stosowane do pomiarów wymagających wysokiej precyzji. \\
    4. Metody analizy danych pomiarowych:

    - Średnia arytmetyczna: Uśrednienie wyników pomiarów w celu zmniejszenia wpływu błędów przypadkowych.

    - Odchylenie standardowe: Miara rozproszenia wyników pomiarów wokół średniej.

    - Niepewność pomiarowa: Określenie zakresu, w którym z określonym prawdopodobieństwem znajduje się prawdziwa wartość mierzonej wielkości. \\
    \textbf{} \\
    Bibliografia: \\
    Halliday, D., Resnick, R., \& Walker, J. (2014). Fundamentals of Physics. Wiley.
    Podstawy fizyki, w tym zagadnienia związane z pomiarami i błędami pomiarowymi.

    PN-EN ISO 3611:2010. Śruby mikrometryczne -- Wymagania i badania.
    Norma dotycząca wymagań i metod badania śrub mikrometrycznych.

    PN-EN ISO 13385-1:2019. Suwmiarki -- Część 1: Wymagania i badania.
    Norma dotycząca wymagań i metod badania suwmiarek.
    }

\textbf{} \\
\centering

\section*{OPIS DOŚWIADCZENIA}

{Celem eksperymentu było porównanie dokładności i precyzji trzech narzędzi pomiarowych (linijka, suwmiarka, śruba mikrometryczna) poprzez wielokrotny pomiar długości trzech przedmiotów (walec, kostka, pierścień). Każdy przedmiot został zmierzony kilka razy za pomocą każdego narzędzia, a wyniki zapisano do dalszej analizy. Po przeprowadzeniu pomiarów obliczono średnie wartości, odchylenia standardowe oraz niepewności pomiarowe, aby ocenić dokładność i precyzję poszczególnych narzędzi, oraz potwierdzić hipotezę tego, że tym większą dokładność ma narzędzie, tym mniejsze będą błędy pomiarowe.}

\section*{OPRACOWANIE WYNIKÓW POMIARÓW}

{W eksperymencie zmierzono długość boków kostki, średnicy zewnętrznej i wewnętrznej pierścienia oraz średnicę i długość walca za pomocą linijki, suwmiarki i śruby mikrometrycznej (mikrometra), a następnie policzono pola/objętości tych przedmiotów. Pomiary zebrano celem porównania dokładności i precyzji narzędzi. Wyniki pomiarów zostały zamieszczone w trzech tabelach jako wstęp do sprawozdania. \\
W przypadku kostki, zmierzone zostały jej boki za pomocą wszystkich trzech narzędzi. W przypadku pierścienia - średnica zewnętrzna i wewnętrzna za pomocą linijki i suwmiarki. Walec zaś był mierzony suwmiarką (długość) oraz mikrometrem (średnica). \\
Podczas pomiarów zauważalnym wnioskiem był fakt tego, że tym niższy błąd pomiarowy narzędzia, tym bardziej dokładne były pomiary. W przypadku kostki obliczone zostały błędy dla pomiaru pośredniego - objętości obiektu. W przypadku linijki z dokładnością wynoszącą $\pm1mm$, odchylenie objętości wynosiło $\pm0.63cm^3$, czyli aż $22.9\%$ od średniej objętości.
Suwmiarka i śruba mikrometryczna te błędy miały już dużo mniejsze. Odpowiednio dla suwmiarki (dokładność $\pm0.05mm$) - $\pm0.03cm^3$, czyli $0.99\%$, oraz dla śruby mikrometrycznej (dokładność $\pm0.01mm$) - $\pm0.02cm^3$, czyli $0.63\%$. \\
Odchylenie liczone ze wzoru $\Delta V = \max(|V - V_{min}|, |V - V_{max}|)$, a procentowe odchylenie - $\frac{\Delta V}{V}*100\%$. \\
W przypadku walca liczone było odchylenie standardowe wzorem $u(h) = \sqrt{\frac{1}{n(n-1)}\sum_{i=1}^{n}(h_i-h_{śr})^2}$ oraz $u(d) = \sqrt{\frac{1}{n(n-1)}\sum_{i=1}^{n}(d_i-d_{śr})^2}$. Odchylenie standardowe objętości walca zostało wyliczone z $u(V) = \sqrt{(\frac{\delta V}{\delta d})^2*u^2(d)+(\frac{\delta V}{\delta h})^2*u^2(h)}$. Wartości te wynosiły odpowiednio: \\$u(d) = 0.0018cm$; $u(h) = 0.0006cm$; $u(V) = 0.0013cm^3$. \\
W przypadku pierścienia podobnie jak w przypadku kostki, policzone zostały pola pierścienia, ich odchylenia, oraz odchylenia procentowe. Pierścień był mierzony linijką oraz suwmiarką. Pole średnie dla linijki wynosi: $S = 1.13cm^2$, odchylenie: $\Delta S = 0.1cm^2$, a odchylenie procentowe: $\delta S = 8.84\%$. \\
Dla suwmiarki zaś, wartości te wynoszą odpowiednio: $S = 1.26cm^2$; $\Delta S = 0.02cm^2$; $\delta S = 1.58\%$}\\


\section*{WNIOSKI}

{Na podstawie wykonanych pomiarów oraz obliczeń, możemy dojść do wniosku, że tym dokładniejsze narzędzie, tym mniejsze odchylenie oraz błąd pomiarowy. Dla narzędzia z największym błędem (linijka), odchylenia procentowe potrafiły dochodzić nawet do $22.9\%$, zaś w przypadku mikrometra i suwmiarki wartości te były dużo mniejsze. Odpowiednio $0.63\%$ i $0.99\%$ w przypadku pomiarów kostki. \\
Na podstawie powyższej analizy można stwierdzić, że większa dokładność narzędzia przekłada się na mniejsze błędy pomiarowe, co potwierdza naszą hipotezę.} \\
\end{document}


%Dodatkowe uwagi:

1. Sprawozdanie może być pisane ręcznie. Proszę jednak o czytelność pisma!!!

2. Sprawozdanie MUSI zawierać wszystkie części (tabela pomiarową, teoria,
przebieg ćwiczenia, obliczenia, niepewności, wnioski i wykresy). Brak
jakiejkolwiek części kwalifikuje do zwrotu złożonego sprawozdania bez dalszego
sprawdzania.

3. Wykresy należy zamieszczać na osobnych kartkach (format A4). Wykonywać za
pomocą komputera lub ręcznie na papierze milimetrowym. Należy tak dobrać
skalę, aby wykres zajmował całą stronę.

4. Punktów pomiarowych naniesionych na wykresach nie łączymy! W przypadku
dopasowania prostej regresji, wraz punktami na wykresie należy nanieść prostą
regresji.

5. Na wykresach razem z punktami należy nanieść niepewności pomiarowe w formie
tzw. krzyży niepewności pomiarowych.

6. Do sprawozdania należy dołączyć kartkę pomiarową z ćwiczenia podpisaną przez
prowadzącego.

7. Przy zapisie wyników wraz z niepewnością obowiązuje zasada podawania 2 cyfr
znaczących (instrukcja ONP).

8. Niepewności pomiarowe w większości przypadków wyliczamy bazując na trzech
metodach:
a) gdy mamy pomiary skorelowane korzystamy z zależności 17 w instrukcji ONP,
b) gdy mamy pomiary nieskorelowane korzystamy z zależności 15 w ONP,
c) w przypadku dopasowywania prostych regresji, niepewności obliczamy ze
wzorów 6 i 7 w ONP.